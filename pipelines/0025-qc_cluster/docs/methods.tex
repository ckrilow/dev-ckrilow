\documentclass{article}

\usepackage[utf8]{inputenc}
\usepackage[T1]{fontenc} % for different characters in author names of cited articles

\usepackage[sort,numbers,authoryear]{natbib}

% maths stuff
\usepackage{amsmath}
\usepackage{mleftright}
\newcommand{\lnb}[1]{%
  \ln\mleft(#1\mright)%
}
\DeclareMathOperator{\Var}{Var}


\begin{document}

\title{Example module 1}
\author{Forename Surname}

\maketitle

\section{My analysis 1a}

Basic qc gene groups All accessed on 3 Aug 2020:
1. Mitochondrial transcripts (56 genes). Downloaded from www.genenames.org/data/genegroup/\#!/group/1972 \\ 
2. Mitochondrial protein coding genes (13 genes). Downloaded from www.genenames.org/data/genegroup/\#!/group/1974 \\ 
3. Ribosomal proteins (164 genes). Downloaded from www.genenames.org/data/genegroup/\#!/group/1054 \\ 
4. Ribosomal RNAs (64 genes). Downloaded from www.genenames.org/data/genegroup/\#!/group/848 


\section{My analysis 1a}

Text describing part a of my analysis 1, as in a paper, citing the relevant methods \citep{article_1,article_2}. 

\section{My analysis 1b}

Text describing part a of my analysis 1, as in a paper, citing the relevant methods \citep{article_3}. 

\section{Note for the future}

Any additional notes!

\bibliographystyle{apalike}
\bibliography{methods-references}

\end{document}
